%%
%    * ----------------------------------------------------------------
%    * "THE BEER-WARE LICENSE" (Revision 42/023):
%    * Moritz Augsburger wrote this file. As long as you retain this notice you
%    * can do whatever you want with this stuff. If we meet some day and you
%    * think this stuff is worth it, you can buy me a beer or a coffee in 
%    * return.
%    * ----------------------------------------------------------------
%

\comment{Mechanik}

\begin{karte}{Beschleunigung -- Kraft}
    \begin{align*}
        F &= m \cdot a \\
        [\text{N} &= \text{kg} \cdot 
            \frac{\text{m}}{\text{s}^2}
        ]
    \end{align*}
\end{karte}

\begin{karte}{Beschleunigung -- Weg}
    \begin{align*}
        x &= \frac{1}{2} \cdot a \cdot t^2 \\
        [
            \text{m} &=
            \frac{\text{m}}{\text{s}^2}
            \cdot \text{s}^2 ]
    \end{align*}
\end{karte}

\begin{karte}{Haftreibung}
    \begin{align*}
        F_\text{H} &= \mu_\text{H} \cdot F_\text{N} \\
    \end{align*}
    \begin{center}
    \begin{tabular}[t]{cl}
        \(\text{F}_\text{H}\) :& Haftreibung\\
        \(\text{\textmu}_\text{H}\) :& Haftreibungskonstante \\
        \(\text{F}_\text{N}\) :& Normalkraft \\
    \end{tabular}
    \end{center}
\end{karte}

\begin{karte}{Gleitreibung}
    \begin{align*}
        F_\text{Gl} &= \mu_\text{Gl} \cdot F_\text{N} \\
    \end{align*}
    \begin{center}
    \begin{tabular}[t]{cl}
        \(\text{F}_\text{Gl}\) :& Gleitreibung\\
        \(\text{\textmu}_\text{Gl}\) :& Gleitreibungskonstante \\
        \(\text{F}_\text{N}\) :& Normalkraft \\
    \end{tabular}
    \end{center}
\end{karte}

\begin{karte}{Haftreibung -- Schiefe Ebene}
    \begin{align*}
        \mu_\text{H} &= \tan \alpha
    \end{align*}
    Winkel \(\alpha\) für gegebenes \( \mu_\text{H} \), ab dem die Haftreibung nicht mehr zum Halten ausreicht, also das Objekt anfängt zu ``rutschen''
\end{karte}

\begin{karte}{Leistung}
    \begin{align*}
        P &= F \cdot v \\
        \left[ \vphantom{\frac{m^2}{s^3}} \right.
            \text{W} &= 
            \text{N} \cdot \frac{\text{m}}{\text{s}} \\
            &= \text{kg} \frac{\text{m}}{\text{s}^2} \cdot \frac{\text{m}}{\text{s}} \\
            &= \text{kg} \frac{\text{m}^2}{\text{s}^3} 
            \left. \vphantom{\frac{m^2}{s^3}} \right]
    \end{align*}
\end{karte}

\begin{karte}{Wirkungsgrad}
    \begin{align*}
        \eta &= \frac{P_\text{out}}{P_\text{in}}
    \end{align*}
\end{karte}

\begin{karte}{Radialbeschleunigung}
    \begin{align*}
        a &= \frac{v^2}{r} \\
        \left[ \vphantom{\frac{\frac{\text{m}^2}{\text{s}^2}}{\text{m}}} \right.
            \frac{\text{m}}{\text{s}^2} &= \left.  \frac{\frac{\text{m}^2}{\text{s}^2}}{\text{m}} 
             \right]
    \end{align*}
\end{karte}

\begin{karte}{Arbeit}
    \begin{align*}
        W &= F \cdot s \\
        \bigg[
            \text{J} &= \text{N} \cdot \text{m} \\
            &= \text{kg}\frac{\text{m}}{\text{s}^2} \cdot \text{m}\\
            &= \text{kg}\frac{\text{m}^2}{\text{s}^2}  
            \bigg]
    \end{align*}
\end{karte}

\begin{karte}{potentielle Energie}
    \begin{align*}
        E_\text{pot} &= m \cdot g \cdot h \\
        \bigg[
            \text{J} &= \text{kg} \cdot \frac{\text{m}}{\text{s}^2} \cdot \text{m} \\
            &= \text{kg}\frac{\text{m}^2}{\text{s}^2} 
            \bigg]
    \end{align*}
\end{karte}

\begin{karte}{kinteische Energie}
    \begin{align*}
        E_\text{kin} &= \frac{1}{2} \cdot m \cdot v^2 \\
        \bigg[
            \text{J} &= \text{kg} \cdot \frac{\text{m}^2}{\text{s}^2} 
            \bigg]
    \end{align*}
\end{karte}

\begin{karte}{Kreisfrequenz}
    \begin{align*}
        \omega &= \frac{2\pi}{T} \\
        \bigg[ \text{s}^{-1} &= \frac{\text{rad}}{\text{s}} \bigg]
    \end{align*}
    \begin{tabular}[t]{cl}
        T:& Kreisfrequenz (Umlaufzeit)
    \end{tabular}
\end{karte}

\begin{karte}{Kreisfrequenz Hook'sche Feder}
    \begin{align*}
        \omega &= \sqrt{\frac{D}{m}} \\
        \bigg[ \text{s}^{-1} &= \sqrt{\frac{\frac{\text{N}}{\text{m}}}{\text{kg}}} \bigg]
    \end{align*}
    \begin{tabular}[t]{cl}
        D:& Federkonstante 
    \end{tabular}
\end{karte}

\begin{karte}{harmonische Schwingung:\\Beschleunigung}
    \begin{align*}
        a(t) &= - \omega^2 \cdot y_0 \cdot \sin \omega t = - \omega^2 \cdot y(t) \\
        \bigg[ \frac{\text{m}}{\text{s}^2} &= \text{s}^{-2} \cdot \text{m} \bigg]
    \end{align*}
\end{karte}

\begin{karte}{harmonische Schwingung:\\Geschwindigkeit}
    \begin{align*}
        v(t) &= \omega \cdot y_0 \cdot \cos \omega t \\
        \bigg[ \frac{\text{m}}{\text{s}} &= \text{s}^{-1} \cdot \text{m} \bigg]
    \end{align*}
\end{karte}

\begin{karte}{harmonische Schwingung:\\Auslenkung}
    \begin{align*}
        y(t) &= y_0 \cdot \sin \omega t
    \end{align*}
\end{karte}

\begin{karte}{potentielle Energie\\Hook'sche Feder}
    \begin{align*}
        W &= \frac{1}{2}\cdot D\cdot x^2 = E_\text{pot} \\
        \bigg[ \text{J} &= \frac{\text{N}}{\text{m}} \text{m}^2 \\
        &= \frac{\text{kg}\frac{\text{m}}{\text{s}^2}}{\text{m}} \cdot \text{m}^2 \\
        &= \text{kg}\frac{\text{m}^2}{\text{s}^2} \bigg]
    \end{align*}
\end{karte}

\begin{karte}{Kraft Hook'sche Feder}
    \begin{align*}
        F &= D \cdot x \\
        \bigg[ \text{N} &= \frac{\text{N}}{\text{m}} \cdot \text{m} \bigg]
    \end{align*}
\end{karte}

\begin{karte}{Inelastischer Stoß}
    \begin{align*}
        v' &= \frac{ m_1v_1+m_2v_2}{m_1+m_2}
    \end{align*}
\end{karte}

\begin{karte}{Elastischer Stoß}
    \begin{align*}
        v_1'&= \frac{(m_1-m_2)v_1 + 2m_2v_2}{m_1+m_2} \\
        v_2'&= \frac{(m_2-m_1)v_2 + 1m_1v_1}{m_2+m_1}
    \end{align*}
\end{karte}

\begin{karte}{Drehimpuls}
    \begin{align*}
        L &= \vartheta \cdot \omega \\
        \bigg[ \text{N m s} &= \text{kg }\text{m}^2 \cdot \text{s}^{-1} \\
            \text{kg} \frac{\text{m}}{\text{s}^2} \text{m s}&= \text{kg} \frac{\text{m}^2}{\text{s}} \\
            \text{kg} \frac{\text{m}^2}{\text{s}} &= \text{kg} \frac{\text{m}^2}{\text{s}}
            \bigg]
    \end{align*}
\end{karte}

\begin{karte}{Kinetische Energie Drehbewegung}
    \begin{align*}
        E_\text{kin} &= \frac{1}{2} \cdot \vartheta \cdot \omega^2 \\
        \bigg[ \text{J} &= \text{kg }\text{m}^2 \cdot \text{s}^{-2} \\
            &= \text{kg} \frac{\text{m}^2}{\text{s}^2} \bigg]
    \end{align*}
\end{karte}

\begin{karte}{Impuls}
    \begin{align*}
        p &= m \cdot v \\
        \bigg[ \frac{\text{kg m}}{\text{s}} &= \text{kg} \cdot \frac{\text{m}}{\text{s}} \bigg]
    \end{align*}
\end{karte}

\begin{karte}{Kreisfrequenz Fadenpendel}
    \begin{align*}
        \omega &= \sqrt{\frac{g}{l}} \\
        \bigg[ \text{s}^{-1} &= \sqrt{ \frac{\text{m}}{\text{s}^2} \cdot \frac{1}{\text{m}} } \\
            &= \sqrt{\text{s}^{-2}} = \text{s}^{-1} \bigg]
    \end{align*}
    Nur bei \(\alpha < 5^\circ\)
\end{karte}

\begin{karte}{Trägheitsmoment Stab um Stabende}
    \begin{align*}
        \vartheta &= \frac{1}{3} \cdot m \cdot l^2 \\
        \bigg[ \text{kg m}^2 &=
            \text{kg} \cdot \text{m}^2 
            \bigg]
    \end{align*}
    \begin{tabular}[t]{cl}
        l:& Länge des homogenen Stabes
    \end{tabular}
\end{karte}

\begin{karte}{Trägheitsmoment Stab um Schwerpunkt}
    \begin{align*}
        \vartheta &= \frac{1}{12} \cdot m \cdot l^2 \\
        \bigg[ \text{kg m}^2 &=
            \text{kg} \cdot \text{m}^2 
            \bigg]
    \end{align*}
    \begin{tabular}[t]{cl}
        l:& Länge des homogenen Stabes
    \end{tabular}
\end{karte}

\begin{karte}{Trägheitsmoment Vollzylinder}
    \begin{align*}
        \vartheta &= \frac{1}{2} \cdot m \cdot r^2 \\
        \bigg[ \text{kg m}^2 &=
            \text{kg} \cdot \text{m}^2 
            \bigg]
    \end{align*}
    \begin{tabular}[t]{cl}
        r:& Durchmesser des Zylinders
    \end{tabular}
\end{karte}

\begin{karte}{Trägheitsmoment Hohlzylinder}
    \begin{align*}
        \vartheta &= m \cdot r^2 \\
        \bigg[ \text{kg m}^2 &=
            \text{kg} \cdot \text{m}^2 
            \bigg]
    \end{align*}
\end{karte}

\begin{karte}{Transformation \\ Geschwindigkeit -- Winkelgeschwindigkeit}
    \begin{align*}
        v &= r \cdot \omega \\
        \bigg[ \frac{\text{m}}{\text{s}} &= 
            \text{m} \cdot \text{s}^{-1}
            \bigg]
    \end{align*}
\end{karte}

\begin{karte}{Trägheitsmoment Kugel}
    \begin{align*}
        \vartheta &= \frac{2}{5} \cdot m \cdot r^2 \\
        \bigg[ \text{kg m}^2 &=
            \text{kg} \cdot \text{m}^2 
            \bigg]
    \end{align*}
\end{karte}

\begin{karte}{leeres Duplikat}
\end{karte}

\begin{karte}{Leistung Translation}
    \begin{align*}
       P &= F \cdot v = M \cdot \omega \\
       \bigg[ \text{W} &= \text{N} \cdot \frac{\text{m}}{\text{s}}  = \text{Nm} \cdot \text{s}^{-1} \\
           \text{kg} \frac{\text{m}^2}{\text{s}^3} &= \text{kg}\frac{\text{m}}{\text{s}^2} \cdot \frac{\text{m}}{\text{s}}
           \bigg]
    \end{align*}
\end{karte}

\begin{karte}{Drehmoment}
    \begin{align*}
        M &= F \cdot r \\
        \bigg[ \text{Nm} &=
            \text{N} \cdot \text{m}
            \bigg]
    \end{align*}
\end{karte}

\begin{karte}{Kreisfrequenz Drehschwingung}
    \begin{align*}
        \omega &= \sqrt{\frac{D}{\vartheta}} \\
        \bigg[ \text{s}^{-1} &=  \sqrt{ \frac{\text{N}}{\text{m}} \cdot \frac{1}{\text{kg m}^2 }}
            \bigg]
    \end{align*}
\end{karte}

\begin{karte}{Rückstellmoment Drehschwingung}
    \begin{align*}
        M &= -D_\varphi \cdot \varphi \\
        [ \text{Nm} &= \text{Nm?} ]
    \end{align*}
    \begin{tabular}[t]{cl}
        \(D_\varphi\) :& Torsionsfederkonstante \\
        \(\varphi\) :& Verdrillungswinkel
    \end{tabular}
\end{karte}

\begin{karte}{Präzessionsfrequenz}
     \begin{align*}
         \omega_\text{p} &= \frac{M}{L} = \frac{ F \cdot r \cdot \sin \varphi }{ \vartheta \cdot \omega_\text{r}} \\
         \bigg[ \text{s}^{-1} &= \frac{\text{Nm}}{\text{N m s}} 
             = \frac{\text{N} \cdot \text{m}}{\text{kg m}^2 \cdot s^{-1}}
            \bigg]
     \end{align*}
\end{karte}

\begin{karte}{Satz von Steiner}
    \begin{align*}
        \vartheta &= m \cdot a^2 + \vartheta_{\text{SP}} \\
        \bigg[ \text{kg m}^2 &=
            \text{m}^2 \cdot \text{kg} + \text{kg m}^2
            \bigg]
    \end{align*}
    \begin{tabular}[t]{cl}
        \(\vartheta_\text{SP}\) & Trägheitsmoment durch Schwerpunkt \\
        \(\vartheta\) & Trägheitsmoment durch neue Achse, \\
        &\(\parallel\) zur Achse von \(\vartheta_\text{SP}\) \\
        a & Abstand der beiden Achsen
    \end{tabular}
\end{karte}

\begin{karte}{Gravitationkonstante}
    \begin{align*}
        \gamma = 6,6742 \cdot 10^{-11} \frac{\text{N m}^2}{\text{kg}^2}
    \end{align*}
\end{karte}

\begin{karte}{Gravitationspotential}
    \begin{align*}
        \varphi &= - \frac{\gamma \cdot m }{r} \\
        \bigg[ \frac{\text{m}^2}{\text{s}^2} &= \frac{\frac{\text{N m}^2}{\text{kg}^2} \cdot \text{kg}}{\text{m}} \\
            &= \text{N}\frac{\text{m}}{\text{kg}} = \text{kg}\frac{\text{m}}{\text{s}^2}\frac{\text{m}}{\text{kg}} 
            \bigg]
    \end{align*}
\end{karte}

\begin{karte}{pot. Energie Gravitation}
    \begin{align*}
        E_\text{pot} &= - \frac{\gamma \cdot m_1 \cdot m_2}{r}\\
        \bigg[ \text{J} &= \frac{\frac{\text{N m}^2}{\text{kg}^2} \cdot \text{kg} \cdot \text{kg}}{\text{m}} \\
            &= \text{Nm} \bigg]
    \end{align*}
\end{karte}

\begin{karte}{Gravitationfeldstärke}
    \begin{align*}
        g &= - \frac{ \gamma \cdot M }{r^2} \\
        \bigg[ \frac{\text{m}}{\text{s}^2} &= \frac{\frac{\text{N m}^2}{\text{kg}^2} \cdot \text{kg}}{\text{m}^2} \\
            &= \frac{\text{N}}{\text{kg}} = \frac{\text{kg}\frac{\text{m}}{\text{s}^2}}{\text{kg}}
            \bigg]
    \end{align*}
    \begin{tabular}[t]{cl}
        M &: Planetenmasse
    \end{tabular}
\end{karte}

\begin{karte}{Gravitationskraft}
    \begin{align*}
        F_\text{G} &= - \gamma \cdot \frac{m_1m_2}{r^2} \\
        \bigg[ N &= \frac{\text{N m}^2}{\text{kg}^2} \cdot \frac{\text{kg}^2}{\text{m}^2} \bigg]
    \end{align*}
\end{karte}

\begin{karte}{Erhaltungssätze der klassischen Physik}
    \begin{itemize}
        \item Energien
        \item Impulse
        \item Drehimpulse
        \item elektrische Ladungen
    \end{itemize}
\end{karte}

\begin{karte}{Corioliskraft}
    \begin{align*}
        F_\text{C} &= m \cdot a_\text{c} = 2 \cdot m \cdot v_\bot \cdot \omega \\
        \bigg[ \text{N} &= \text{kg} \cdot \frac{\text{m}}{\text{s}^2} = \text{kg} \cdot \frac{\text{m}}{\text{s}} \cdot \text{s}^{-1} \bigg]
    \end{align*}
    \begin{tabular}[t]{cl}
        \(\text{a}_\text{c}\): & Coriolisbeschleunigung \\
        \(v_\bot\): & Geschwindigkeit des Körpers, rel. \\
            & zum rotierenden Bezugssystem \\
        \(\omega\): & Winkelgeschwindigkeit Bezugssystem
    \end{tabular}
\end{karte}

\begin{karte}{Keplersche Gesetze}
    \begin{itemize}
        \item Planeten auf Ellipsen mit Sonne im gemeinsamen Brennpunkt
        \item Radiusvektor überstreicht in gleicher Zeit gleiche Fläche: \(\frac{\Delta A}{\Delta t} = \text{const}\)
        \item Umlaufzeit \(T_{1,2}\), große Halbachse \(a_{1,2}\) zweier Planeten: \( \frac{T_1^2}{T_2^2} = \frac{a_1^3}{a_2^3} \)
    \end{itemize}
\end{karte}

\begin{karte}{Planet auf Kreisbahn}
    \begin{align*}
        \frac{r_\text{p}^3}{T_\text{p}^2} &= \gamma \frac{m_\text{s}}{4 \pi^2} = const.
    \end{align*}
    \begin{tabular}[t]{cl}
        \(r_\text{p}\): & Radius Planetenbahn \\
        \(T_\text{p}\): & Umlaufzeit Planet \\
        \(m_\text{s}\): & Masse der Sonne \\
    \end{tabular}
\end{karte}

\begin{karte}{Gebundener und ungebundener Zustand}
    \begin{align*}
        E &= E_\text{kin} + E_\text{pot} = \frac{1}{2} m_2 v^2  - \gamma\frac{m_1  m_2}{r}\\
    \end{align*}
    \begin{tabular}[t]{cl}
        \(E \ge 0\): & ungebunder Zustand, \(m_2\) kann sich  \\
            & beliebig weit von \(m_1\) entfernen\\
        \(E < 0\): & gebunder Zustand
    \end{tabular}
\end{karte}
