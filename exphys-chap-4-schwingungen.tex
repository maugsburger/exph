%%
%    * ----------------------------------------------------------------
%    * "THE BEER-WARE LICENSE" (Revision 42/023):
%    * Moritz Augsburger wrote this file. As long as you retain this notice you
%    * can do whatever you want with this stuff. If we meet some day and you
%    * think this stuff is worth it, you can buy me a beer or a coffee in 
%    * return.
%    * ----------------------------------------------------------------
%

\comment{Schwingungen}

\begin{karte}{Bewegungsgleichung \\ harmonischer Oszillator}
    \begin{align*}
        m \ddot x + Dx &= 0
    \end{align*}
\end{karte}

\begin{karte}{Bewegungsgleichung \\ freier, gedämpfter Oszillator}
    \begin{align*}
        m \ddot x + &\beta \dot x + Dx = 0 \\
%        \text{mit } &\beta = -6 \pi \eta r
    \end{align*}
\end{karte}

\begin{karte}{harmonischer Oszillator}
    \begin{align*}
        x(t) &= x_0 \cos \left( \omega_0 t \right) + \frac{v_0}{\omega_0} \sin \left( \omega_0 t \right) \\
%        \text{mit } \omega_0 &= \sqrt{\frac{D}{m}}
    \end{align*}
\end{karte}

\begin{karte}{gedämpfter Oszillator}
    \begin{align*}
        y(t) &= y_0 \mathrm e^{-\delta t} \sin \left( \sqrt{  \omega_0^2 - \delta^2 } \cdot t + \varphi_0 \right)
    \end{align*}
\end{karte}

\begin{karte}{gedämpfte Schwingung \\ Reibung Stokesche Kugel}
    \begin{align*}
        \beta &=  -6 \pi \eta r
    \end{align*}
\end{karte}


\begin{karte}{Kreisfrequenz physikalisches Pendel}
    \begin{align*}
        \omega_0 &= \sqrt{ \frac{mgS}{\vartheta} }
    \end{align*}
\end{karte}

\begin{karte}{Bewegungsgleichung\\erzwungene Schwingung}
    \begin{align*}
         m \ddot x + \beta \dot x + Dx &= D L_0 \sin \left( \omega t \right) 
    \end{align*}
\end{karte}

\begin{karte}{erzwungene, gedämpfte Schwingung}
    \begin{align*}
        \frac{x_0}{L_0} &= \frac{ \omega_0^2 }{\sqrt{ \left( \omega_0^2 - \omega^2 \right)^2 - \left( 2 \delta \omega \right) ^2   }}
    \end{align*}
    \begin{tabular}[t]{cl}
        \(x_0 \): & Amplitude der Schwingung \\
        \(L_0 \): & Amplitude des Erregers \\
        \(\omega_0 \): & Eigenfrequenz der Schwingung \\
        \(\omega \): & Frequenz des Erregers \\
        \(\delta \): & Dämpfung \\
    \end{tabular}
\end{karte}

\begin{karte}{Schwebungsfreqzenz \\ schwache Kopplung}
    \begin{align*}
        \omega_{schwebung} &= \omega_1 - \omega_2
    \end{align*}
    \begin{tabular}[t]{cl}
        \( D_{12}\): & Kopplungsfeder (\(D_{12} \ll D \)) \\
        \( D\): & Randfeder \\
        \( m\): & \( m_{Pendel2} = m_{Pendel1} \)
    \end{tabular}
\end{karte}

\begin{karte}{gleichpasige, gekoppelte Schwingung}
    \begin{align*}
        \omega_1 &= \omega_0 = \sqrt{\frac{D}{m}}
    \end{align*}
    \begin{tabular}[t]{cl}
        \( D_{12}\): & Kopplungsfeder (immer entspannt) \\
        \( D\): & Randfeder \\
        \( m\): & \( m_{Pendel2} = m_{Pendel1} \)
    \end{tabular}
\end{karte}

\begin{karte}{gegenphasige, gekoppelte Schwingung}
    \begin{align*}
        \omega_2 &= \sqrt{ \frac{D +2 \cdot D_{12}}{m} } \\
                 &= \sqrt{ \omega_0 + 2 \frac{D_{12}}{m} }
    \end{align*}
    \begin{tabular}[t]{cl}
        \( D_{12}\): & Kopplungsfeder \\
        \( D\): & Randfeder \\
        \( m\): & \( m_{Pendel2} = m_{Pendel1} \)
    \end{tabular}
\end{karte}

\comment{Wellen}

\begin{karte}{Wellengleichung}
    \begin{align*}
        y(t,x) &= y_0 \sin (\omega t - k x)
    \end{align*}
\end{karte}

\begin{karte}{Schallgeschwindigkeit in Stab}
    \begin{align*}
        c &= \sqrt{ \frac{E}{\rho} }
    \end{align*}
    \begin{tabular}[t]{cl}
        \( E \): & Elastizitätsmodul \\
        \( \rho \): & Dichte \\
    \end{tabular}
\end{karte}

\begin{karte}{Dopplereffekt\\ bewegte Quelle}
    \begin{align*}
        f' &= \frac{ f_0 }{ 1 - \frac {v}{c} }
    \end{align*}
    \begin{tabular}[t]{cl}
        \( f' \): & empfangen Frequenz \\
        \( f \): & gesendete Frequenz \\
        \( v \): & Geschwindigkeit Quelle\\
        \( c \): & Schallgeschwindigkeit \\
    \end{tabular}
\end{karte}

\begin{karte}{Dopplereffekt\\ bewegter Beobachter}
    \begin{align*}
        f' &= f_0 ( 1 + \frac {v}{c} )
    \end{align*}
    \begin{tabular}[t]{cl}
        \( f' \): & empfangen Frequenz \\
        \( f \): & gesendete Frequenz \\
        \( v \): & Geschwindigkeit Beobachter\\
        \( c \): & Schallgeschwindigkeit \\
    \end{tabular}
\end{karte}

\begin{karte}{Wellenzahl}
    \begin{align*}
        k &= \frac{2 \pi}{\lambda}
    \end{align*}
    \begin{tabular}[t]{cl}
        \( \lambda \): & Wellenlänge \\
    \end{tabular}
\end{karte}

\begin{karte}{Phasengeschwindigkeit}
    \begin{align*}
        c &= \frac{\lambda}{T} = \frac{\omega}{k}
    \end{align*}
    \begin{tabular}[t]{cl}
        \( T = \tfrac {2\pi}{\omega}\): & ``Ort'' um \(\lambda\) gewandert \\
        \( k \): &  Wellenzahl \\
        \( \lambda \): & Wellenlänge \\
    \end{tabular}
\end{karte}
