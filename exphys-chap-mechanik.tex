%%
%    * ----------------------------------------------------------------
%    * "THE BEER-WARE LICENSE" (Revision 42/023):
%    * Moritz Augsburger wrote this file. As long as you retain this notice you
%    * can do whatever you want with this stuff. If we meet some day and you
%    * think this stuff is worth it, you can buy me a beer or a coffee in 
%    * return.
%    * ----------------------------------------------------------------
%

\comment{Mechanik}

\begin{karte}{Beschleunigung -- Weg}
    \begin{align*}
        F &= m \cdot a \\
        [\text{N} &= \text{kg} \cdot 
            \frac{\text{m}}{\text{s}^2}
        ]
    \end{align*}
\end{karte}

\begin{karte}{Beschleunigung -- Kraft}
    \begin{align*}
        x &= \frac{1}{2} \cdot a \cdot t^2 \\
        [
            \text{m} &=
            \frac{\text{m}}{\text{s}^2}
            \cdot \text{s}^2 ]
    \end{align*}
\end{karte}

\begin{karte}{Haftreibung}
    \begin{align*}
        F_H &= \mu_H \cdot F_N \\
    \end{align*}
    \begin{center}
    \begin{tabular}[t]{cl}
        \(\text{F}_H\) :& Haftreibung\\
        \(\text{\textmu}_H\) :& Haftreibungskonstante \\
        \(\text{F}_N\) :& Normalkraft \\
    \end{tabular}
    \end{center}
\end{karte}

\begin{karte}{Gleitreibung}
    \begin{align*}
        F_{Gl} &= \mu_{Gl} \cdot F_N \\
    \end{align*}
    \begin{center}
    \begin{tabular}[t]{cl}
        \(\text{F}_{Gl}\) :& Gleitreibung\\
        \(\text{\textmu}_{Gl}\) :& Gleitreibungskonstante \\
        \(\text{F}_N\) :& Normalkraft \\
    \end{tabular}
    \end{center}
\end{karte}

\begin{karte}{Haftreibung -- Schiefe Ebene}
    \begin{align*}
        \mu_H &= \tan \alpha
    \end{align*}
\end{karte}

\begin{karte}{Leistung}
    \begin{align*}
        P &= F \cdot v \\
        \left[ \vphantom{\frac{m^2}{s^3}} \right.
            \text{W} &= 
            \text{N} \cdot \frac{\text{m}}{\text{s}} \\
            &= \text{kg} \frac{\text{m}}{\text{s}^2} \cdot \frac{\text{m}}{\text{s}} \\
            &= \text{kg} \frac{\text{m}^2}{\text{s}^3} 
            \left. \vphantom{\frac{m^2}{s^3}} \right]
    \end{align*}
\end{karte}

\begin{karte}{Wirkungsgrad}
    \begin{align*}
        \eta &= \frac{P_{out}}{P_{in}}
    \end{align*}
\end{karte}

\begin{karte}{Radialbeschleunigung}
    \begin{align*}
        a &= \frac{v^2}{r} \\
        \left[ \vphantom{\frac{\frac{\text{m}^2}{\text{s}^2}}{\text{m}}} \right.
            \frac{\text{m}}{\text{s}^2} &= \left.  \frac{\frac{\text{m}^2}{\text{s}^2}}{\text{m}} 
             \right]
    \end{align*}
\end{karte}

\begin{karte}{Arbeit}
    \begin{align*}
        W &= F \cdot s \\
        \bigg[
            \text{J} &= \text{N} \cdot \text{m} \\
            &= \text{kg}\frac{\text{m}}{\text{s}^2} \cdot \text{m}\\
            &= \text{kg}\frac{\text{m}^2}{\text{s}^2}  
            \bigg]
    \end{align*}
\end{karte}

\begin{karte}{potentielle Energie}
    \begin{align*}
        E_{pot} &= m \cdot g \cdot h \\
        \bigg[
            \text{J} &= \text{kg} \cdot \frac{\text{m}}{\text{s}^2} \cdot \text{m} \\
            &= \text{kg}\frac{\text{m}^2}{\text{s}^2} 
            \bigg]
    \end{align*}
\end{karte}

\begin{karte}{kinteische Energie}
    \begin{align*}
        E_{kin} &= \frac{1}{2} \cdot m \cdot v^2 \\
        \bigg[
            \text{J} &= \text{kg} \cdot \frac{\text{m}^2}{\text{s}^2} 
            \bigg]
    \end{align*}
\end{karte}

\begin{karte}{Kreisfrequenz}
    \begin{align*}
        \omega &= \frac{2\pi}{T} \\
        \bigg[ \text{s}^{-1} &= \frac{\text{rad}}{\text{s}} \bigg]
    \end{align*}
    \begin{tabular}[t]{cl}
        T:& Kreisfrequenz (Umlaufzeit)
    \end{tabular}
\end{karte}

\begin{karte}{Kreisfrequenz Hook'sche Feder}
    \begin{align*}
        \omega &= \sqrt{\frac{D}{m}} \\
        \bigg[ \text{s}^{-1} &= \sqrt{\frac{\frac{\text{N}}{\text{m}}}{\text{kg}}} \bigg]
    \end{align*}
    \begin{tabular}[t]{cl}
        D:& Federkonstante 
    \end{tabular}
\end{karte}

\begin{karte}{harmonische Schwingung:\\Beschleunigung}
    \begin{align*}
        a(t) &= - \omega^2 \cdot y_0 \cdot \sin \omega t = - \omega^2 \cdot y(t) \\
        \bigg[ \frac{\text{m}}{\text{s}^2} &= \text{s}^{-2} \cdot \text{m} \bigg]
    \end{align*}
\end{karte}

\begin{karte}{harmonische Schwingung:\\Geschwindigkeit}
    \begin{align*}
        v(t) &= \omega \cdot y_0 \cdot \cos \omega t \\
        \bigg[ \frac{\text{m}}{\text{s}} &= \text{s}^{-1} \cdot \text{m} \bigg]
    \end{align*}
\end{karte}

\begin{karte}{harmonische Schwingung:\\Auslenkung}
    \begin{align*}
        y(t) &= y_0 \cdot \sin \omega t
    \end{align*}
\end{karte}

\begin{karte}{potentielle Energie\\Hook'sche Feder}
    \begin{align*}
        W &= \frac{1}{2}\cdot D\cdot x^2 = E_{pot} \\
        \bigg[ \text{J} &= \frac{\text{N}}{\text{m}} \text{m}^2 \\
        &= \frac{\text{kg}\frac{\text{m}}{\text{s}^2}}{\text{m}} \cdot \text{m}^2 \\
        &= \text{kg}\frac{\text{m}^2}{\text{s}^2} \bigg]
    \end{align*}
\end{karte}

\begin{karte}{Kraft Hook'sche Feder}
    \begin{align*}
        F &= D \cdot x \\
        \bigg[ \text{N} &= \frac{\text{N}}{\text{m}} \cdot \text{m} \bigg]
    \end{align*}
\end{karte}

\begin{karte}{Inelastischer Stoß}
    \begin{align*}
        v' &= \frac{ m_1v_1+m_2v_2}{m_1+m_2}
    \end{align*}
\end{karte}

\begin{karte}{Elastischer Stoß}
    \begin{align*}
        v_1'&= \frac{(m_1-m_2)v_1 + 2m_2v_2}{m_1+m_2} \\
        v_2'&= \frac{(m_2-m_1)v_2 + 1m_1v_1}{m_2+m_1}
    \end{align*}
\end{karte}

% TODO: Ab hier Einheiten!

\begin{karte}{Drehimpuls}
    \begin{align*}
        L &= \vartheta \cdot \omega
    \end{align*}
\end{karte}

\begin{karte}{Kinetische Energie Drehbewegung}
    \begin{align*}
        E_{kin} &= \frac{1}{2} \cdot \vartheta \cdot \omega^2
    \end{align*}
\end{karte}

\begin{karte}{Impuls}
    \begin{align*}
        p &= m \cdot v
    \end{align*}
\end{karte}

\begin{karte}{Kreisfrequenz Fadenpendel}
    \begin{align*}
        \omega &= \sqrt{\frac{g}{l}}
    \end{align*}
    Nur bei \(\alpha < 5^\circ\)
\end{karte}

\begin{karte}{Trägheitsmoment Stab um Schwerpunkt}
    \begin{align*}
        \vartheta &= \frac{1}{12} \cdot m \cdot L^2
    \end{align*}
\end{karte}

\begin{karte}{Trägheitsmoment Vollzylinder}
    \begin{align*}
        \vartheta &= \frac{1}{2} \cdot m \cdot r^2
    \end{align*}
\end{karte}

\begin{karte}{Trägheitsmoment Hohlzylinder}
    \begin{align*}
        \vartheta &= m \cdot r^2
    \end{align*}
\end{karte}

\begin{karte}{Transformation \\ Geschwindigkeit -- Winkelgeschwindigkeit}
    \begin{align*}
        v &= r \cdot \omega
    \end{align*}
\end{karte}

\begin{karte}{Trägheitsmoment Kugel}
    \begin{align*}
        \vartheta &= \frac{2}{5} \cdot m \cdot r^2
    \end{align*}
\end{karte}

\begin{karte}{Trägheitsmoment Stab um Stabende}
    \begin{align*}
        \vartheta &= \frac{1}{3} \cdot m \cdot L^2
    \end{align*}
\end{karte}

\begin{karte}{Leistung Translation}
    \begin{align*}
       P &= F \cdot v = M \cdot \omega
    \end{align*}
\end{karte}

\begin{karte}{Drehmoment}
    \begin{align*}
        M &= F \cdot r
    \end{align*}
\end{karte}

\begin{karte}{Kreisfrequenz Drehschwingung}
    \begin{align*}
        w &= \sqrt{\frac{D}{\vartheta}}
    \end{align*}
\end{karte}

\begin{karte}{Rückstellmoment Drehschwingung}
    \begin{align*}
        M &= -D \cdot \varphi
    \end{align*}
\end{karte}

\begin{karte}{Präzessionsfrequenz}
    \begin{align*}
        \omega_p &= \frac{M}{L} = \frac{ F \cdot r \cdot \sin \varphi }{ \vartheta \cdot \omega_r}
    \end{align*}
\end{karte}

\begin{karte}{Satz von Steiner}
    \begin{align*}
        \vartheta &= m \cdot a^2 + \vartheta_{SP}
    \end{align*}
\end{karte}

\begin{karte}{Gravitationspotential}
    \begin{align*}
        \varphi &= - \frac{\gamma \cdot m }{r}
    \end{align*}
\end{karte}

\begin{karte}{pot. Energie Gravitation}
    \begin{align*}
        E_{pot} &= - \frac{\gamma \cdot m_1 \cdot m_2}{r}
    \end{align*}
\end{karte}

\begin{karte}{Gravitationfeldstärke}
    \begin{align*}
        g &= - \frac{ \gamma \cdot M }{r^2}
    \end{align*}
\end{karte}

\begin{karte}{Gravitationskraft}
    \begin{align*}
        F_G &= - \gamma \frac{m_1m_2}{r^1}
    \end{align*}
\end{karte}

\begin{karte}{Erhaltungssätze der klassischen Physik}
    \begin{itemize}
        \item Energien
        \item Impulse
        \item Drehimpulse
        \item elektrische Ladungen
    \end{itemize}
\end{karte}

\begin{karte}{Corioliskraft}
    \begin{align*}
        F_C &= m \cdot a_c = 2 \cdot m \cdot v_\bot \cdot \omega
    \end{align*}
\end{karte}

\begin{karte}{Keplersche Gesetze}
    \begin{itemize}
        \item Planeten auf Ellipsen mit Sonne im Brennpunkt
        \item gleiche Zeit - gleiche Fläche
        \item \(\frac{T_{Umlauf}^2}{r_{Bahn}^3} = const. \)
    \end{itemize}
\end{karte}

\begin{karte}{Planeten}
    \begin{align*}
        \frac{r_p^3}{T_p^2} &= \gamma \frac{m_s}{4 \pi^2} = const.
    \end{align*}
\end{karte}

