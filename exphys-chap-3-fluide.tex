%%
%    * ----------------------------------------------------------------
%    * "THE BEER-WARE LICENSE" (Revision 42/023):
%    * Moritz Augsburger wrote this file. As long as you retain this notice you
%    * can do whatever you want with this stuff. If we meet some day and you
%    * think this stuff is worth it, you can buy me a beer or a coffee in 
%    * return.
%    * ----------------------------------------------------------------
%

\comment{Fluide}

\begin{karte}{Viskosität\\ ``Zähigkeit''}
    \begin{align*}
        \eta \left[ \si{\newton\second\per\square\meter} \right]
    \end{align*}
\end{karte}

\begin{karte}{Dichte}
    \begin{align*}
        \varrho \left[ \si{\kilogram\per\cubic\meter} \right]
    \end{align*}
\end{karte}

\begin{karte}{Oberflächenspannung}
    \begin{align*}
        \sigma \left[ \si{\joule\per\square\meter} \right]
    \end{align*}
\end{karte}

\begin{karte}{hydrostatischer Druck \\ Schweredruck}
    \begin{align*}
        p(h) &= p_0 + \varrho \cdot h \cdot g \\
        \bigg[ \si{\pascal} &= \si{\pascal} + \underbrace
            {\frac{\si{\kilogram}}{\si{\cubic\meter}} \cdot \si{\meter} \cdot \frac{\si{\meter}}{\si{\square\second}}}
            _{ \frac{\si{\kilogram}}{\si{\meter\square\second}} \cdot \frac{\si{\meter}}{\si{\meter}}=
            \frac{\si{\newton}}{\si{\square\meter}} = \si{\pascal}}
            \bigg]
    \end{align*}
    \begin{tabular}[t]{cl}
        \(p_0\): & (Luft-)Druck an der Oberfläche \\
        \(h\): & Tiefe
    \end{tabular}
\end{karte}

\begin{karte}{Auftrieb}
    \begin{align*}
        F &= (\varrho_\text{Fl} - \varrho_\text{K}) \cdot V_\text{K} \cdot g \\
        \bigg[ \si{\newton} &= \frac{\si{\kilogram}}{\si{\cubic\meter}} \cdot \si{\cubic\meter} \cdot \frac{\si{\meter}}{\si{\square\second}} = 
            \si{\kilogram} \frac{\si{\meter}}{\si{\square\second}} \bigg]
    \end{align*}
    \begin{tabular}[t]{cl}
        \( \varrho_\text{Fl} < \varrho_\text{K} \Leftrightarrow F_\text{A} < F_\text{G} \Longrightarrow \) & Körper sinkt \\
        \( \varrho_\text{Fl} = \varrho_\text{K} \Leftrightarrow F_\text{A} = F_\text{G} \Longrightarrow \) & Körper schwebt \\
        \( \varrho_\text{Fl} > \varrho_\text{K} \Leftrightarrow F_\text{A} > F_\text{G} \Longrightarrow \) & Körper steigt
    \end{tabular}
\end{karte}

\begin{karte}{Barometrische Höhenformel}
    \begin{align*}
        p &= p_0 \cdot \exp \left(-\frac{\varrho_0}{p_0}\cdot g \cdot h \right)  
    \end{align*}
\end{karte}

\begin{karte}{Rückstellkraft Oberflächenspannung}
    \begin{align*}
        F &= 2 \cdot \sigma \cdot l \\
        \bigg[ \si{\newton} &= \frac{\si{\joule}}{\si{\square\meter}} \cdot \si{\meter}  = \frac{\si{\newton}}{\si{\meter}} \cdot \si{\meter}  \bigg]
    \end{align*}
    \begin{tabular}[t]{cl}
        \(\sigma\): & Oberflächenspannung \\
        \(l\): & Länge der Randlinie des Bügels
    \end{tabular}
\end{karte}

\begin{karte}{Oberflächenenergie}
    \begin{align*}
        W &= A \cdot \sigma \\
        \bigg[ \si{\joule} &= \si{\square\meter} \cdot \frac{\si{\joule}}{\si{\square\meter}} \bigg] \\
    \end{align*}
\end{karte}

\begin{karte}{Druck in Flüssigkeitskugel}
    \begin{align*}
        p &= 2\frac{\sigma}{r} \text{~~Vollkugel (Wassertropfen)}\\
        p &= 3\frac{\sigma}{r} \text{~~Hohlkugel (Seifenblase)} \\
        \Bigg[ \si{\pascal} &= \frac{\frac{\si{\joule}}{\si{\square\meter}}}{\si{\meter}}  = \frac{\frac{\si{\newton\meter}}{\si{\square\meter}}}{\si{\meter}}  = \frac{\si{\newton}}{\si{\square\meter}} \Bigg]
    \end{align*}
\end{karte}

\begin{karte}{Kugeloberfläche- und Volumen}[Geometrie]
    \begin{align*}
        A &= 4\pi r^2 &&\text{Kugeloberfläche} \\
        A &= \frac{4}{3}\pi r^3 &&\text{Kugelvolumen}
    \end{align*}
\end{karte}

\begin{karte}{Kontinuitätsgleichung\\für inkompressible Medien}
    \begin{align*}
        A_1 v_1 &= A_2 v_2 \\
        \text{für } \varrho &= \text{const}
    \end{align*}
\end{karte}

\begin{karte}{Bernoulli-Gleichung}
    \begin{align*}
        \underbrace{\frac{\varrho}{2}v_1^2}_\text{Staudruck} + \underbrace{p_1}_\text{stat. Druck} &= \underbrace{p_0}_\text{Gesamtdruck}
    \end{align*}
\end{karte}

\begin{karte}{Newtonsches Reibungsgesetz \\  Viskosität zwischen Platten}
    \begin{align*}
        F  &=  \eta \cdot A \cdot \frac{\mathrm dv}{\mathrm dx} \\
        \bigg[ \si{\newton} &= \frac{\si{\newton\second}}{\si{\square\meter}} \cdot \si{\square\meter} \cdot \frac{\si{\meter\per\second}}{\si{\meter}} 
            \bigg]
    \end{align*}
\end{karte}

\begin{karte}{Geschwindigkeit im Stromröhrchen}
    \begin{align*}
        v(r) &= \frac{p_1-p_2}{4\eta l}(R^2-r^2) \\
        \Bigg[ \si{\meter\per\second} &= \frac{\si{\pascal}}{\si{\newton\second\per\square\meter}\si{\meter}} \si{\square\meter} = 
             \frac{\si{\newton\per\square\meter}}{\si{\newton\second\per\square\meter}\si{\meter}} \si{\square\meter} =
             \si{\square\meter\per\meter\per\second}
            \Bigg] 
    \end{align*}
     \begin{tabular}[t]{cl}
         \(p_{1,2}\): &Druck vor und hinter dem Röhrchen \\
         \(R\): &Radius des umschließenden Rohres \\
         \(r\): &Radius des Röhrchens
     \end{tabular}
\end{karte}

\begin{karte}{Antriebskraft Rohrströmung}
    \begin{align*}
        F &= \pi \cdot r^2 \cdot  \Delta p \\
        \bigg[ \si{\newton} &= \si{\square\meter} \cdot \si{\pascal} = \si{\square\meter} \cdot \si{\newton\per\square\meter} \bigg]
    \end{align*}
\end{karte}

\begin{karte}{Gesetz von Hagen-Poiseuille}
    \begin{align*}
        \dot M &= \frac{\varrho\cdot\pi}{8\cdot\eta} \cdot \frac{ \Delta p}{l} \cdot R^4 \sim R^4 \\
        \Bigg[ \si{\kilogram\per\second} &= 
            \frac{\si{\kilogram\per\cubic\meter}}{\si{\newton\second\per\square\meter}} \cdot \frac{\si{\newton\per\square\meter}}{\si{\meter}} \cdot \si{\raiseto{4}\meter} = \si[sticky-per=true]{\newton\kilogram\raiseto{6}\meter \per \newton\second\raiseto{6}\meter}
            \Bigg]
    \end{align*}
     \begin{tabular}[t]{cl}
         \(\dot M\): &Massenstromstärke \\
         \(\Delta p\): &Druckdifferenz vor und hinter dem Rohr \\
         \(R\): &Radius des Rohres 
     \end{tabular}
\end{karte}

\begin{karte}{Stockesches Gesetz für Kugel}
    \begin{align*}
        F_\text{R} &= 6 \cdot \pi \cdot \eta \cdot r \cdot v \\
        \bigg[ \si{\newton} &= \si{\newton\second\per\square\meter} \cdot \si{\meter} \cdot\si{\meter\per\second} \bigg] \\
    \end{align*}
    \begin{align*}
        v = \text{const für:}& \\
        mg - \left| F_\text{A} \right| &= 6 \cdot \pi \cdot \eta \cdot r \cdot v = F_\text{R}
    \end{align*}
\end{karte}

\begin{karte}{Reynolds-Zahl}
    \begin{align*}
        \mathit{Re} &= \frac{\varrho \cdot L \cdot v}{\eta} \\
        \Bigg[ 1 &= \frac{\si{\kilogram\per\cubic\meter} \cdot \si{\meter} \cdot \si{\meter\per\second}}{\si{\newton\second\per\square\meter}} 
            = \frac{\si{\kilogram\per\second\per\meter}}{\si{\kilogram \per\second\per\meter }}
            \Bigg]
    \end{align*}
    Sobald \(\mathit{Re}\) einen bestimmten Grenzwert über\-schreitet (z.B. 2300 bei Rohrströmung), schlägt die Strömung von laminar in turbulent um.
\end{karte}

\begin{karte}{Luftwiderstand}
    \begin{align*}
        F &=  c_\mathrm w \cdot \frac{\varrho}{2} \cdot v^2 \cdot A \\
        \bigg[ \si{\newton} &= 1 \cdot \si{\kilogram\per\cubic\meter} \cdot \si{\square\meter\per\square\second} \cdot \si{\square\meter} \bigg]
    \end{align*}
     \begin{tabular}[t]{cl}
         \(c_\mathrm w\):& Strömungswiderstandskoeffizient \\
         \(A\): &Stirnfläche
     \end{tabular}

\end{karte}

