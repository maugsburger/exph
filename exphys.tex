%%
%    * ----------------------------------------------------------------
%    * "THE BEER-WARE LICENSE" (Revision 42/023):
%    * Moritz Augsburger wrote this file. As long as you retain this notice you
%    * can do whatever you want with this stuff. If we meet some day and you
%    * think this stuff is worth it, you can buy me a beer or a coffee in 
%    * return.
%    * ----------------------------------------------------------------
%
\documentclass[a8paper,9pt,print,grid=front]{kartei}
%\documentclass[a8paper,9pt,grid=front]{kartei}

\usepackage[utf8]{inputenc}
\usepackage{mathtools}
\usepackage{textcomp}

\begin{document}
\fach{Physik}
\comment{Mechanik}

\begin{karte}{Beschleunigung -- Weg}
    \begin{align*}
        F &= m \cdot a \\
        [\text{N} &= \text{kg} \cdot 
            \frac{\text{m}}{\text{s}^2}
        ]
    \end{align*}
\end{karte}

\begin{karte}{Beschleunigung -- Kraft}
    \begin{align*}
        x &= \frac{1}{2} \cdot a \cdot t^2 \\
        [
            \text{m} &=
            \frac{\text{m}}{\text{s}^2}
            \cdot \text{s}^2 ]
    \end{align*}
\end{karte}

\begin{karte}{Haftreibung}
    \begin{align*}
        F_H &= \mu_H \cdot F_N \\
    \end{align*}
    \begin{center}
    \begin{tabular}[t]{cl}
        \(\text{F}_H\) :& Haftreibung\\
        \(\text{\textmu}_H\) :& Haftreibungskonstante \\
        \(\text{F}_N\) :& Normalkraft \\
    \end{tabular}
    \end{center}
\end{karte}

\begin{karte}{Gleitreibung}
    \begin{align*}
        F_{Gl} &= \mu_{Gl} \cdot F_N \\
    \end{align*}
    \begin{center}
    \begin{tabular}[t]{cl}
        \(\text{F}_{Gl}\) :& Gleitreibung\\
        \(\text{\textmu}_{Gl}\) :& Gleitreibungskonstante \\
        \(\text{F}_N\) :& Normalkraft \\
    \end{tabular}
    \end{center}
\end{karte}

\begin{karte}{Haftreibung -- Schiefe Ebene}
    \begin{align*}
        \mu_H &= tan \alpha
    \end{align*}
\end{karte}

\begin{karte}{Leistung}
    \begin{align*}
        P &= F \cdot v \\
        \left[ \vphantom{\frac{m^2}{s^3}} \right.
            \text{W} &= 
            \text{N} \cdot \frac{\text{m}}{\text{s}} \\
            &= \text{kg} \frac{\text{m}}{\text{s}^2} \cdot \frac{\text{m}}{\text{s}} \\
            &= \text{kg} \frac{\text{m}^2}{\text{s}^3} 
            \left. \vphantom{\frac{m^2}{s^3}} \right]
    \end{align*}
\end{karte}

\begin{karte}{Wirkungsgrad}
    \begin{align*}
        \eta &= \frac{P_{out}}{P_{in}}
    \end{align*}
\end{karte}

\begin{karte}{Radialbeschleunigung}
    \begin{align*}
        a &= \frac{v^2}{r} \\
        \left[ \vphantom{\frac{\frac{\text{m}^2}{\text{s}^2}}{\text{m}}} \right.
            \frac{\text{m}}{\text{s}^2} &= \left.  \frac{\frac{\text{m}^2}{\text{s}^2}}{\text{m}} 
             \right]
    \end{align*}
\end{karte}

\begin{karte}{Arbeit}
    \begin{align*}
        W &= F \cdot s \\
        \bigg[
            \text{J} &= \text{N} \cdot \text{m} \\
            &= \text{kg}\frac{\text{m}}{\text{s}^2} \cdot \text{m}\\
            &= \text{kg}\frac{\text{m}^2}{\text{s}^2}  
            \bigg]
    \end{align*}
\end{karte}

\begin{karte}{potentielle Energie}
    \begin{align*}
        E_{pot} &= m \cdot g \cdot h \\
        \bigg[
            \text{J} &= \text{kg} \cdot \frac{\text{m}}{\text{s}^2} \cdot \text{m} \\
            &= \text{kg}\frac{\text{m}^2}{\text{s}^2} 
            \bigg]
    \end{align*}
\end{karte}

\begin{karte}{kinteische Energie}
    \begin{align*}
        E_{kin} &= \frac{1}{2} \cdot m \cdot v^2 \\
        \bigg[
            \text{J} &= \text{kg} \cdot \frac{\text{m}^2}{\text{s}^2} 
            \bigg]
    \end{align*}
\end{karte}

\begin{karte}{}
    \begin{align*}
        &=
    \end{align*}
\end{karte}

\begin{karte}{}
    \begin{align*}
        &=
    \end{align*}
\end{karte}

\begin{karte}{}
    \begin{align*}
        &=
    \end{align*}
\end{karte}

\begin{karte}{}
    \begin{align*}
        &=
    \end{align*}
\end{karte}

\begin{karte}{}
    \begin{align*}
        &=
    \end{align*}
\end{karte}

\begin{karte}{}
    \begin{align*}
        &=
    \end{align*}
\end{karte}

\begin{karte}{}
    \begin{align*}
        &=
    \end{align*}
\end{karte}

\begin{karte}{}
    \begin{align*}
        &=
    \end{align*}
\end{karte}

\end{document}
