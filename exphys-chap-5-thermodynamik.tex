%%
%    * ----------------------------------------------------------------
%    * "THE BEER-WARE LICENSE" (Revision 42/023):
%    * Moritz Augsburger wrote this file. As long as you retain this notice you
%    * can do whatever you want with this stuff. If we meet some day and you
%    * think this stuff is worth it, you can buy me a beer or a coffee in 
%    * return.
%    * ----------------------------------------------------------------
%

\comment{Thermodynamik}

\begin{karte}{0. Hauptsatz der Thermodynamik}
    Zwei Körper im thermischen Gleichgewicht haben die selbe Temperatur
\end{karte}

\begin{karte}{1. Hauptsatz der Thermodynamik}
    Es ist unmöglich, Energie aus dem nichts zu gewinnen. \\
    Ein perpetuum mobule erster Art ist unmöglich.
\end{karte}

\begin{karte}{2. Hauptsatz der Thermodynamik}
    Wärmeenergie fließt von selbst immer nur zum kälteren Körper, aber nie umgekehrt. \\
    Ein perpetuum mobule erster Art ist unmöglich.
\end{karte}

\begin{karte}{3. Hauptsatz der Thermodynamik}
    Am absoluten Nullpunkt ist die Entropie 0. \\
    Es ist unmöglich, diesen zu erreichen.
\end{karte}

\begin{karte}{Längenausdehnung}
    \begin{align*}
        \frac{\Delta l}{l} &= \alpha \cdot \Delta T
    \end{align*}
    \begin{tabular}[t]{cl}
        \( l \): & Länge \\
        \( \Delta l \): & Längenänderung \\
        \( \Delta T \): & Temperaturänderung \\
        \( \alpha \): & Wärmeausdehnungskoeffizient
    \end{tabular}
\end{karte}

\begin{karte}{Volumenausdehnung}
    \begin{align*}
        \frac{\Delta V}{V} &= \gamma \cdot \Delta T
    \end{align*}
    \begin{tabular}[t]{cl}
        \( V \): & Volumen \\
        \( \Delta V \): & Volumenänderung \\
        \( \Delta T \): & Temperaturänderung \\
        \( \gamma \): & Volumenausdehnungskoeffizient
    \end{tabular}
\end{karte}

\begin{karte}{Volumenausdehnungskoeffizient \\ Festkörper}
    \begin{align*}
        \gamma &= 3 \cdot \alpha
    \end{align*}
    \begin{tabular}[t]{cl}
        \( \gamma \): & Volumenausdehnungskoeffizient \\
        \( \alpha \): & Wärmeausdehnungskoeffizient
    \end{tabular}
\end{karte}

\begin{karte}{spezifische Wärme, \\ Wärmekapazität}
    \begin{align*}
        \Delta Q&= c \cdot m \cdot \Delta T = C \cdot \Delta T \\
        C &= c \cdot m
    \end{align*}
    \begin{tabular}[t]{cl}
        \( \Delta Q \): & Wärmeenergie \\
        \( \Delta T \): & Temperaturänderung \\
        \( c \): & spezifische Wärme \( [ \si{\joule\per\kilogram\per\kelvin} ] \) \\
        \( C \): & Wärmekapazität \( [ \si{\joule\per\kelvin} ] \) \\
    \end{tabular}
\end{karte}

\begin{karte}{ideale Gasgleichung}
    \begin{align*}
        p \cdot V &= n \cdot R \cdot T = N \cdot k \cdot T
    \end{align*}
    \begin{tabular}[t]{cl}
        \( p \): & Druck \\
        \( V \): & Volumen \\
        \( n \): & Stoffmenge \\
        \( R \): & universelle Gaskonstante \([ \si{\joule\per\mol\per\kelvin} ] \)\\
        \( T \): & Temperatur in Kelvin\\
        \( N \): & Teilchenzahl \\
        \( k = \tfrac{R}{N_A} \): & Boltzmann-Konstante 
    \end{tabular}
\end{karte}

\begin{karte}{Teilchenzahl}
    \begin{align*}
        N &= n \cdot N_A
    \end{align*}
    \begin{tabular}[t]{cl}
        \( n \): & Stoffmenge \\
        \( N_A \): & Avogadro-Konstante \\
        & \( 6,022045 \cdot 10^{23} \tfrac{Teilchen}{mol}\)
    \end{tabular}
\end{karte}

\begin{karte}{Wärmebillanz Zustandsänderung}
    \begin{align*}
        \Delta U &= \Delta Q \cdot \Delta W
    \end{align*}
    \begin{tabular}[t]{cl}
        \( \Delta U \): & innere Energie\\
        \( \Delta Q\): & Wärmeenergie \\
        \( \Delta W \): & mechanische Arbeit \\
    \end{tabular}
\end{karte}

\begin{karte}{innere Energie}
    \begin{align*}
        \Delta U &= \tfrac{f}{2}\cdot n \cdot  R \cdot  T \Delta T
    \end{align*}
    \begin{tabular}[t]{cl}
        \( f \): & Freiheitsgrade Teilchen \\
        \( n \): & Stoffmenge \\
        \( R \): & universelle Gaskonstante \([ \si{\joule\per\mol\per\kelvin} ] \)\\
        \( T \): & Temperatur in Kelvin\\
    \end{tabular}
\end{karte}

\begin{karte}{Freiheitsgrade}
    \begin{tabular}[t]{rc}
        einatomiges Gas & \( f = 3 \) \\
        zweiatomiges Gas & \( f = 5 \) \\
        Atom in Festkörper & \( f = 6 \)
    \end{tabular}
\end{karte}

\begin{karte}{Boltzmann-Konstante}
    \begin{align*}
        k &= \frac{R}{N_A} = 1,3807 \cdot 10^{-23} \si{\joule\per\kelvin}
    \end{align*}
    \begin{tabular}[t]{cl}
        \( R \): & universelle Gaskonstante \([ \si{\joule\per\mol\per\kelvin} ] \)\\
        \( N_A \): & Avogadro-Konstante \([ \si{\per\mol} ] \)\\
    \end{tabular}
\end{karte}

\begin{karte}{Zugeführte Wärmeenergie\\isochor, isobar}
    \begin{align*}
        \Delta Q &= n \cdot C_{v,p} \cdot \Delta T \\
        C_v &= \tfrac{f}{2} R && \text{(isochor)} \\
        C_p &= \tfrac{f+2}{2} R && \text{(isobar)} \\
    \end{align*}
\end{karte}

\begin{karte}{Adiabatenkoeffizient}
    \begin{align*}
        \kappa &= \frac{f+2}{f} = \frac{C_p}{C_v}
    \end{align*}
\end{karte}

\begin{karte}{Isotherme Energieänderung}
    \begin{align*}
        \Delta W &= n \cdot R \cdot T \cdot \ln \frac{V_2}{V_1} = - \Delta Q
    \end{align*}
    \begin{tabular}[t]{cl}
        \( n \): & Stoffmenge \\
        \( R \): & universelle Gaskonstante \([ \si{\joule\per\mol\per\kelvin} ] \)\\
        \( T \): & Temperatur in Kelvin\\
        \( V_1 \): & Volumen \(t_0\)\\
        \( V_2 \): & Volumen \(t_1\)\\
    \end{tabular}
\end{karte}

